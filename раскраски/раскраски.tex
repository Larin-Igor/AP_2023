\documentclass[12 pt, a4paper]{article}% тип документа, размер шрифта
\usepackage{cmap}	
\usepackage{hyperref}
\hypersetup{
	colorlinks=true,
	linkcolor=blue,
	urlcolor=blue,
}
\usepackage{mathtext}
\usepackage[T2A]{fontenc}%поддержка кириллицы в ЛаТеХ
\usepackage[utf8]{inputenc}%кодировка
\usepackage[english,russian]{babel}
\usepackage{indentfirst}

\usepackage{amsmath,amsfonts,amssymb,amsthm,mathtools} % AMS
\usepackage{amsmath}%удобная вёрстка многострочных формул, масштабирующийся текст в формулах, формулы в рамках и др.
\usepackage{amsfonts}%поддержка ажурного и готического шрифтов — например, для записи символа {\displaystyle \mathbb {R} } \mathbb {R} 
\usepackage{amssymb}%amsfonts + несколько сотен дополнительных математических символов
\frenchspacing%запрет длинного пробела после точки
\usepackage{setspace}%возможность установки межстрочного интервала
\usepackage{indentfirst}%пакет позволяет делать в первом абзаце после заголовка абзацный отступ
\onehalfspacing%установка полуторного интервала по умолчанию
\usepackage{graphicx}%подключение рисунков
\graphicspath{{images/}}%путь ко всем рисункам
\usepackage{caption}
\usepackage{float}%плавающие картинки
\usepackage{tikz} % это для чудо-миллиметровки
\usepackage[export]{adjustbox}
\usepackage{pgfplots}%для построения графиков
\pgfplotsset{compat=newest, y label style={rotate=-90},  width=10 cm}%версия пакета построения графиков, ширина графиков
\usepackage{pgfplotstable}%простое рисование табличек
\usepackage{lastpage}%пакет нумерации страниц
\usepackage{comment}%возможность вставлять большие комменты
\usepackage{float}
%%%%% ПОЛЯ
\setlength\parindent{0pt} 
\usepackage[top = 2 cm, bottom = 2 cm, left = 1.5 cm, right = 2 cm]{geometry}
\setlength\parindent{0pt}
%%%%% КОЛОНТИТУЛЫ
\usepackage[shortlabels]{enumitem}

\usepackage{array,tabularx,tabulary,booktabs} % Дополнительная работа с таблицами
\usepackage{longtable} % Длинные таблицы
\usepackage{multirow} % Слияние строк в таблице
\usepackage{colortbl} % Цветная заливка в таблице
\usepackage[labelsep=period,labelfont=rm,tablename={Таблица},tablewithin=none]{caption}
\usepackage{makecell} 
\usepackage{ctable} % for \specialrule command 

\usepackage{fancybox, fancyhdr}
\pagestyle{fancy} 
\fancyhead[L]{\textit{6 класс}}
\fancyhead[C]{\textit{{ЛМШ "Алые паруса" 2023}}}
\fancyhead[R]{\textit{11 июня}} % ЛЮБАЯ ДОПОЛНИТЕЛЬНАЯ ИНФОРМАЦИЯ
%\fancyfoot[R]{Задание с двух сторон!}
\renewcommand{\footrulewidth}{0.3 mm}

\usepackage{tikzsymbols}
\usepackage{textcomp}
\usepackage{parskip}
\usepackage{graphicx}
\graphicspath{{pictures/}}
\DeclareGraphicsExtensions{.pdf,.png,.jpg}
\usepackage{wrapfig}
%%% Заголовок

%%% Новые команды
\newcommand{\z}[1]{{{\vspace{0.6cm} \large\textbf{{Задача {#1}} \\ }}}}
\newcommand{\task}[1]{{{\vspace{0.6cm} \vspace{-2ex} \textbf{№{#1}}  }}}
\newcommand{\otv}{{\vspace{0.3cm} \textbf{Решение: } \\}}
\newcommand{\uk}{\underline{\textit{Указание.}} }
\newcommand{\opr}{\textit{Определение: }}
\newcommand{\sol}[1]{{{\vspace{0.3cm} \textbf{{Задача {#1}} }\\ }}}
\newcommand{\RomanNumeralCaps}[1]
{\MakeUppercase{\romannumeral #1}}

\usepackage{cancel}
\usepackage{epigraph} 
\setlength\parindent{0pt}
\setlength\parskip{1ex plus 2pt minus 1pt}
\newcommand\X{\par\noindent---~}
\usepackage{ upgreek }
\begin{document} % конец преамбулы, начало документа
	\newpage
	\begin{flushright}
		\textit{<<Все мы - связанный граф, но сможем ли мы когда-нибудь стать полным графом...>>}
	\end{flushright}
	\begin{figure}[t]
		\begin{minipage}[h]{0.33\linewidth}
			\includegraphics[width=0.33\linewidth, left]{logo.jpg}
		\end{minipage}
		%%	\hfill
		\begin{minipage}[h]{0.33\linewidth}
			\centering
			\large{\textbf{ГРАФЫ}}\\
		\end{minipage}
		\begin{minipage}[h]{0.33\linewidth}
			\includegraphics[width=0.33\linewidth, right]{logo.jpg}
		\end{minipage}
		\label{ris:image1}
	\end{figure}
	
	\begin{comment}
		\begin{figure}[b]
			\begin{minipage}[h]{0.33\linewidth}
				\includegraphics[width=0.33\linewidth, left]{logo.jpg}
			\end{minipage}
			%%\hfill
			\begin{minipage}[h]{0.33\linewidth}
				\centering
				\large{\textbf{Удачи \Winkey}}
			\end{minipage}
			\begin{minipage}[h]{0.33\linewidth}
				\includegraphics[width=0.33\linewidth, right]{logo.jpg}
			\end{minipage}
			\label{ris:image1}
		\end{figure}
		
		\begin{tabular}{lcr}
			\includegraphics[width=0.2\linewidth]{logo.jpg} &
			\vspace{-2ex}
			\large\textbf{Вступительная работа} &
			\includegraphics[width=0.2\linewidth]{logo.jpg}
		\end{tabular}
	\end{comment}
	
	\large
	\raggedright
	\textbf{\textit{Граф}} - это конечное множество точек на плоскости, некоторые
	пары которых соединены линиями. Эти точки называются вершинами, а линии -
	ребрами графа.\\
	\textbf{\textit{Степень вершины}} - это количество ребер выходящих из вершины графа.

	\task{1} В графе с \cancel{5} n вершинами любые две вершины соединены ребром. Сколько
	всего рёбер в этом графе?\\
	\task{2а} Можно ли расставить цифры от 1 до 9 по кругу так, чтобы сумма никаких двух
	соседних чисел не делилась ни на 3, ни на 5, ни на 7?\\
	\vspace{-1ex}
	\task{2б}Можно ли записать цифры от 0 до 9 в строку так, чтобы число, составленное
	из любых двух подряд идущих цифр, делилось на 7 или на 13?\\
	
	\task{3}Каждые 5 минут случайные два призрака отправляют детей возрождаться.
	 Оказалось, что каждый отправил детей возрождаться ровно три раза. Могло ли так оказаться, что к тому моменту прошло ровно 895 минут?
	
	\task{4} Как связаны сумма степеней вершин и кол-во рёбер?\\
	
	\task{5a} \textbf{Лемма о рукопожатиях.} Докажите, что чётно число людей, которые в своей
	жизни сделали нечётное число рукопожатиий.\\
	
	\task{5б} Верно ли, что число вершин нечётной степени любого графа чётно?\\
	
	\task{6} В некой компании из шести человек любые двое либо знакомы друг с другом, либо не
	знакомы. Докажите, что среди этих шести человек найдутся трое попарно знакомых или трое попарно незнакомых.\\
	
	\task{7} Сколько ребер куба (максимум) можно перекусить посередине, чтобы
	он не распался на части?\\
	
	\textit{\textbf{Эйлеровым графом}} называется граф, в котором есть \textit{эйлеров цикл}.\\
	\textbf{\textit{Эйлеров цикл}} - это путь, проходящий по всем рёбрам графа и притом только по одному разу\\
	
	\newpage
	
	\task{8а} Через город Кенигсберг протекает река, в русле которой расположены два
	острова. С большего острова ведет по два моста на каждый из
	берегов и один мост на меньший остров. Кроме этого
	моста с меньшего острова ведет по одному мосту на
	каждый из берегов. Некто хочет совершить прогулку по
	городу, пройдя по каждому мосту ровно один раз. Удастся ли ему это?\\
	\task{8б} Можно ли нарисовать изображенные на рисунках граф не отрывая карандаш от бумаги и проводя каждое ребро ровно один раз?\\
	\task{8в} Какое условие для графа должно выполняться, чтобы он был эйлеровым.\\
	\begin{center}
			\begin{figure}[h]
			\includegraphics[width=\linewidth]{pic.png}
		\end{figure}
	\end{center}
	\vspace{-6ex}
	\task{9а} \textbf{В следующих задачах нужно восстановить исходный граф.} Вначале был граф с пятью вершинами. Для каждой из пяти вершин нарисовали,
	какой граф останется после удаления этой вершины вместе с исходящими из неё рёбрами. Вы видите
	справа пять получившихся графов, каждый в отдельной рамке.\\
	\task{9б} В графе известны степени всех вершин: 6, 5, 3, 3, 2, 2, 1. \\
	\task{9в} В графе известны степени всех вершин: 5, 5, 4, 4, 4, 1, 1. \\
	\begin{center}
			\begin{figure}[h]
				\centering
			\includegraphics[width=0.33\linewidth]{pic2.png}
		\end{figure}
	\end{center}

	\newpage
	\task{0} В любом ли графе о двух или более вершинах найдутся две вершины одинаковой степени?\\
	
	\task{0} В R-ской Городской Думе 17 депутатов, из них любые двое либо дружат, либо враждуют, либо равно-
	душны друг к другу. Докажите, что среди R-ских депутатов найдутся трое попарно дружащих или трое попарно враждующих,	или трое попарно равнодушных друг к другу.\\
	\task{0} На доске был нарисован граф с n вершинами. Петя пронумеровал вершины натуральными числами от 1 до n
	одним способом, а Вася — другим. Оказалось, что какие два номера x и y ни возьми, выполнено ровно одно из двух: ребром
	соединены либо две вершины с Петиными номерами x и y, либо две вершины с Васиными номерами x и y. Найдите хотя бы
	один пример графа, который мог быть на доске, если а) n = 4; б) n = 5\\
	
	\task{0} В стране Тили-Мили-Трямдии 2021 города. Некоторые города
	соединены дорогами. При этом для любой пары городов существует ровно один
	«маршрут», по которому можно проехать из одного города в другой. Сколько
	всего дорог в Тили-Мили-Трямдии?\\
	\task{0} В шахматном турнире участвовали семь школьников. Известно, что
	Миша сыграл 6 партий, Коля – 5, Илья и Гриша – по три, Андрей и Сева – по две,
	а Максим – одну. С кем сыграл Илья?\\
	\task{0} В некоторой стране целых четыре столицы и много других городов. Некоторые города соединены двусторонними
	авиалиниями, причём из каждой столицы выходит по 179 авиалиний, а из остальных городов — по 42. Президент страны хочет,чтобы из каждой столицы можно было долететь до любой другой напрямую или с пересадками в других городах. Докажите, что, как бы ни летали самолёты, желание президента либо уже выполнено, либо может быть выполнено, если запустить всего одну новую авиалинию.\\
	\task{0} В стране из каждого города выходит 100 дорог и от каждого города
	можно добраться до любого другого. Одну дорогу закрыли на ремонт.
	Докажите, что и теперь от каждого города можно добраться до любого другого.\\
	\task{0} В стране 360 городов, некоторые из них соединены двусторонними авиалиниями. Из каждого города выходит
	179 или 180 линий, причём оба варианта встречаются. Верно ли, что из любого города можно долететь до любого другого,
	сделав не более двух пересадок?\\
	\task{0} Пешеход обошёл все улицы одного города, пройдя каждую ровно два
	раза, но не смог обойти их, пройдя каждую лишь один раз. Могло ли такое быть?\\
	\task{0}Сколько ребер нужно нарисовать в графе с 6 вершинами, чтобы
	треугольник(и) не был(и) подграфом(ми) этого графа\\
\end{document} 