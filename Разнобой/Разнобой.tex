\documentclass[12 pt, a5paper]{article}% тип документа, размер шрифта
\usepackage{cmap}  
\usepackage{hyperref}
\hypersetup{
	colorlinks=true,
	linkcolor=blue,
	urlcolor=blue,
}
\usepackage{mathtext}
\usepackage[T2A]{fontenc}%поддержка кириллицы в ЛаТеХ
\usepackage[utf8]{inputenc}%кодировка
\usepackage[english,russian]{babel}
\usepackage{indentfirst}

\usepackage{amsmath,amsfonts,amssymb,amsthm,mathtools} % AMS
\usepackage{amsmath}%удобная вёрстка многострочных формул, масштабирующийся текст в формулах, формулы в рамках и др.
\usepackage{amsfonts}%поддержка ажурного и готического шрифтов — например, для записи символа {\displaystyle \mathbb {R} } \mathbb {R} 
\usepackage{amssymb}%amsfonts + несколько сотен дополнительных математических символов
\frenchspacing%запрет длинного пробела после точки
\usepackage{setspace}%возможность установки межстрочного интервала
\usepackage{indentfirst}%пакет позволяет делать в первом абзаце после заголовка абзацный отступ
\onehalfspacing%установка полуторного интервала по умолчанию
\usepackage{graphicx}%подключение рисунков
\graphicspath{{images/}}%путь ко всем рисункам
\usepackage{caption}
\usepackage{float}%плавающие картинки
\usepackage{tikz} % это для чудо-миллиметровки
\usepackage[export]{adjustbox}
\usepackage{pgfplots}%для построения графиков
\pgfplotsset{compat=newest, y label style={rotate=-90},  width=10 cm}%версия пакета построения графиков, ширина графиков
\usepackage{pgfplotstable}%простое рисование табличек
\usepackage{lastpage}%пакет нумерации страниц
\usepackage{comment}%возможность вставлять большие комменты
\usepackage{float}
%%%%% ПОЛЯ
\setlength\parindent{0pt} 
\usepackage[top = 2 cm, bottom = 2 cm, left = 1.5 cm, right = 2 cm]{geometry}
\setlength\parindent{0pt}
%%%%% КОЛОНТИТУЛЫ
\usepackage[shortlabels]{enumitem}

\usepackage{array,tabularx,tabulary,booktabs} % Дополнительная работа с таблицами
\usepackage{longtable} % Длинные таблицы
\usepackage{multirow} % Слияние строк в таблице
\usepackage{colortbl} % Цветная заливка в таблице
\usepackage[labelsep=period,labelfont=rm,tablename={Таблица},tablewithin=none]{caption}
\usepackage{makecell} 
\usepackage{ctable} % for \specialrule command 

\usepackage{fancybox, fancyhdr}
\pagestyle{fancy} 
\fancyhead[L]{\textit{6 класс}}
\fancyhead[C]{\textit{{ЛМШ "Алые паруса"}}}
\fancyhead[R]{\textit{the second day}} % ЛЮБАЯ ДОПОЛНИТЕЛЬНАЯ ИНФОРМАЦИЯ
%\fancyfoot[R]{Задание с двух сторон!}
\renewcommand{\footrulewidth}{0.3 mm}

\usepackage{tikzsymbols}
\usepackage{textcomp}
\usepackage{parskip}
\usepackage{graphicx}
\graphicspath{{pictures/}}
\DeclareGraphicsExtensions{.pdf,.png,.jpg}
\usepackage{wrapfig}
%%% Заголовок

%%% Новые команды
\newcommand{\z}[1]{{{\vspace{0.6cm} \large\textbf{{Задача {#1}} \\ }}}}
\newcommand{\task}[1]{{{\vspace{0.6cm} \vspace{-2ex} \textbf{№{#1}}  }}}
\newcommand{\otv}{{\vspace{0.3cm} \textbf{Решение: } \\}}
\newcommand{\uk}{\underline{\textit{Указание.}} }
\newcommand{\opr}{\textit{Определение: }}
\newcommand{\sol}[1]{{{\vspace{0.3cm} \textbf{{Задача {#1}} }\\ }}}
\newcommand{\RomanNumeralCaps}[1]
{\MakeUppercase{\romannumeral #1}}

\usepackage{cancel}
\setlength\parindent{0pt}
\setlength\parskip{1ex plus 2pt minus 1pt}
\newcommand\X{\par\noindent---~}
\usepackage{ upgreek }
\begin{document} % конец преамбулы, начало документа
	\newpage
	\begin{figure}[t]
	\begin{minipage}[h]{0.33\linewidth}
		\includegraphics[width=0.2\linewidth, left]{logo.jpg}
	\end{minipage}
	%%	\hfill
	\begin{minipage}[h]{0.33\linewidth}
		\centering
		\large{\textbf{Разнобой}}\\
	\end{minipage}
	\begin{minipage}[h]{0.32\linewidth}
		\includegraphics[width=0.2\linewidth, right]{logo.jpg}
	\end{minipage}
	\label{ris:image1}
\end{figure}
	
	\begin{flushright}     
		\textit{<<Может ли  Всемогущий Бог создать камень, \\
			который он сам поднять не может?>>}
	\end{flushright}
	\begin{comment}
	\begin{figure}[b]
		\begin{minipage}[h]{0.5\linewidth}
			\includegraphics[width=0.2\linewidth, left]{logo.jpg}
		\end{minipage}
		\hfill
		\begin{minipage}[h]{0.5\linewidth}
			\includegraphics[width=0.2\linewidth, right]{logo.jpg}
		\end{minipage}
		\label{ris:image1}
	\end{figure}
	\end{comment}
	
	\begin{comment}
		\begin{tabular}{lcr}
			\includegraphics[width=0.2\linewidth]{logo.jpg} &
			\vspace{-2ex}
			\large\textbf{Разнобой} &
			\includegraphics[width=0.2\linewidth]{logo.jpg}
		\end{tabular}
	\end{comment}
	\centering
	\raggedright
	\task{0} Сложите 3 карандаша так, чтобы их носики не касались стола (конструкция должна быть устойчивой).
	\\
	\task{1} Как разделить поровну семь одинаковых пирожных между 12 гостями так, чтобы каждое пирожное было
	разделено не более, чем на 5 частей?
	\\
	\task{2} Как с помощью чашечных весов и гирь весом 1, 3, 9, 27, и 81 грамм (использовать каждую можно только один
	раз) уравновесить груз весом
	\\ \hspace{1em} a) 31 грамм
	\\  \hspace{1em} б) 32 грамма
	\\  \hspace{1em} в) 49 грамм
	\\
	\task{3}
	Студенты, находясь на отдыхе, разрезали арбуз на 4 части и съели. Могло ли получится 5 корок?
	\\
	\task{4} Три курицы за три дня несут три яйца. Сколько яиц снесут 12 таких же курей за 12 дней?
	\\
	\task{5} Разрежьте квадрат $7 \times 7$ на пять частей и переложите их так, чтобы получились 3 квадрата: $2 \times 2$,  $3 \times 3$ и $6 \times 6$. Постарайтесь сделать это несколькими способами.
\end{document}